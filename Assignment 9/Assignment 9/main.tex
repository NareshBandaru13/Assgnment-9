\documentclass{beamer}
\usetheme{CambridgeUS}

\setbeamertemplate{caption}[numbered]{}

\usepackage{enumitem}
\usepackage{amsmath}
\usepackage{amssymb}
\usepackage{gensymb}
\usepackage{graphicx}
\usepackage{txfonts}

\def\inputGnumericTable{}

\usepackage[latin1]{inputenc}                                 
\usepackage{color}                                            
\usepackage{array}                                            
\usepackage{longtable}                                        
\usepackage{calc}                                             
\usepackage{multirow}                                         
\usepackage{hhline}                                           
\usepackage{ifthen}
\usepackage{caption}

\title{AI1110 \\ Assignment 9}
\author{Bandaru Naresh Kumar \\ AI21BTECH11006}
\date{}
\begin{document}
	% The title page
	\begin{frame}
		\titlepage
	\end{frame}
	
	% The table of contents
	\begin{frame}{Outline}
    		\tableofcontents
	\end{frame}
	
	% The question
	\section{Question}
	\begin{frame}{Exercise 10.25}
       If $R_n(\tau) = N\delta(\tau)$ and\\
       $x(t) = A\cos\omega_0t + n(t)$\\
       $H(\omega) = \dfrac{1}{\alpha+j\omega}$\\
       $y(t) = B\cos(\omega_0 +t +\phi) + y_n(t)$\\
       where $y_n(t)$ is the component of the output y(t) due to n(t), find the value of $\alpha$ that maximises the signal to noise ratio $\dfrac{B^2}{E(y_n^2(t))}$
	\end{frame}
	
	% The solution
	\section{Solution}
	\begin{frame}{Solution}
	    We have,\\
	    \begin{align}
	    B &= A|H(\omega_0)| = \dfrac{A}{\sqrt{\alpha^2+\omega_0^2}}\\
	    S_{y_n}(\omega) &= \dfrac{N}{\alpha^2+\omega_0^2}\\
	    R_{y_n}(\tau) &= \dfrac{N}{2\alpha} e^{-\alpha|\tau|}\\
	    E{y_n^2(t)} &= R_{y_n}(0) = \dfrac{N}{2\alpha}
	    \end{align}
	 \end{frame}
	 
	 % The final answer
	\section{Answer}
	\begin{frame}{Answer}
	   Hence,\\
	    $\dfrac{B^2}{E(y_n^2(t))} = \dfrac{2A^2}{N}\dfrac{\alpha}{\alpha^2+\omega_0^2}$\\
	Differentiating, we get
	\begin{align}
	\dfrac{1(\alpha^2+\omega_0^2)-\alpha(2\alpha)}{(\alpha^2+\omega_0^2)^2} &= 0\\
	\omega_0^2 - \alpha^2 &= 0\\
	\alpha &= \omega_0 
	\end{align}
	Also,$f^{\prime\prime}(\alpha)<0$\\
	$\therefore\alpha = \omega_0$ is the maxima value for given             ratio.
	\end{frame}
\end{document}